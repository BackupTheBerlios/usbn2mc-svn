%% Erl�uterungen zu den Befehlen erfolgen unter
%% diesem Beispiel.                
\documentclass{book}
\usepackage[ansinew]{inputenc}
\usepackage[T1]{fontenc} 
\usepackage{ngerman}
\usepackage{graphicx}
\usepackage{listings}

\title{usbprojects-net.de}
\author{Benedikt Sauter}
\begin{document}

\maketitle
\tableofcontents
\newpage
\chapter{USB}
	\section{Grundkurs}
	\section{Ger�te-Stack}
	\section{Host-Stack}
	\section{Debuggen}
	\section{Bausteine}
\chapter{Computer Treiber}
	\section{Linux}
	\section{Windows}
	\section{Mac}
\chapter{Tutorial}
\chapter{Projekte}
	\section{Logik Analysator}
	\section{CDC Class: virtuelle USB Netzwerkkarte}
	\section{HID Class: PS2 zu USB Wandler}
	\section{usbprog, AVR Atmel USB Programmieradapter}
	\section{vPort, universeller Schnittstellenwandler}
	\section{Firmware bequem �ber das Internet und USB updaten}

\chapter{Literatur}


\end{document}
