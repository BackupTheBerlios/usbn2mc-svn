\documentclass[11pt]{article}

% TODO
%   - "Linux Inside" Stempel


% CV and Expertise

%\usepackage{a4}
%\usepackage{lscape}
\usepackage{fancybox,graphicx,url,german}

\evensidemargin=0cm
\oddsidemargin=0cm
\hoffset=-0.5cm
\topmargin=-2cm
\textwidth=29.7cm
\textheight=21cm

\pagestyle{empty}
\parindent=0pt

%\pdfpageheight=\textheight
%\pdfpagewidth=\textwidth

\begin{document}


% ========= PAGE 2 ===============
\begin{minipage}[t]{7cm}
\hfill {\Large\sf 2}\\
\hrule
\shadowbox{{\sf Das Problem}}

\bigskip

Aktuelle Computer ("`PC"') sind oft 

\begin{itemize}
\item zu teuer (etwa 1000 Euro)
\item zu laut 
\item zu leistungsf"ahig f"ur allt"agliche Anwendungen
\item zu hohe Stromverbraucher (200 Watt und mehr)
\item zu gross, schwer und sperrig
\end{itemize}

\medskip

\shadowbox{{\sf Die L"osung}}

\begin{center}
  {\large Einen passenden Rechner selberbauen!}
\end{center}

Vorteile 

\begin{itemize}
\item macht Spass
\item man lernt was
\item man kann verschiedene preiswerte Teile kombinieren
\item man kann den Rechner nach eigenem Geschmack bauen
\end{itemize}


% \fbox{Just an fbox test}

\end{minipage}
\hskip 12mm
\rule[-18.8cm]{0.2pt}{5mm}  % tickmark
\hskip 12mm
% ========= PAGE 3 ===============
\begin{minipage}[t]{7cm}
\hfill {\Large\sf 3}\\
\hrule

\shadowbox{{\sf Beispiel}}

\begin{itemize}
\item Rechnerplatine ({\em "`Mainboard"'}).
      Via EPIA Mini-ITX {\bf ME-6000} oder "ahnlich (600 MHz, 25 Watt).
      \url{http://www.mini-itx.de}, \url{http://www.mini-itx.com},
      120 EUR 

\item Arbeitsspeicher ({\em "`RAM"'}),
      30 bis 50 EUR  (512 MByte)

\item Festplatte, z.B. Samsung SP1614N (160 GByte),
      80 EUR 

\item CDROM/DVD Laufwerk, 20 EUR

\item Tastatur, 20 EUR

\item Maus, 20 EUR

\item Geh"ause (mit Netzteil), 50 bis 80 EUR,
  \url{http://www.mini-itx.de} und \url{http://www.mini-itx.com}

\end{itemize}

% Die Eigenschaften

% \begin{itemize}

% \item Mit 600 MHz Taktfrequenz ausreichend schnell (h"angt vom
%   Hauptspeicher ab)

% \item Kleine Abmessungen XXX

% \item Geringe Leistungsaufnahme (ca. 20 bis 40 Watt)

% \end{itemize}

Alle Teile kann man bei \url{http://www.mini-itx.de} bestellen. Auch
einige Augsburger Gesch"afte wie Arlt (\url{http://www.arlt.com}) und 
K\&M (\url{http://www.kmshop.de}) haben die meisten Teile.

Insgesamt kommt man auf ca. 300 bis 350 Euro Materialkosten (ohne
Bildschirm).


\vskip 5mm


\end{minipage}
\hskip 12mm
\rule[-18.8cm]{0.2pt}{5mm}  % tickmark
\hskip 12mm
% ========= PAGE 4 ===============
\begin{minipage}[t]{7cm}
\hfill {\Large\sf 4}\\
\hrule
\shadowbox{{\sf Projekte}}

Hier ist das Handbuch zum Mainboard hilfreich.

\begin{itemize}
\item Speicherriegel mit Gef"uhl in die Rechnerplatine stecken.

\item Festplatte an das "`breite Kabel"' (IDE Flachbandkabel) und an die
  Stromversorgung stecken. Jumper auf {\em Master}.

\item CDROM-Laufwerk wie die Festplatte anschliessen. Auf dem breiten
  Kabel sind dazu zwei Anschl"usse vorgesehen. Jumper auf {\em Slave}.

\item Die Bedienelemente der Frontplatte anschliessen, mindestens den
  Ein-/Ausschalter. 

\item Den weissen Netzteilstecker in die Rechnerplatine stecken (die vielen bunten Kabel).

\end{itemize}

\shadowbox{{\sf Einschalten}}

\begin{itemize}
  
\item Direkt nach dem Einschalten die {\tt Entfernen}-Taste dr"ucken
  und damit in das BIOS wechseln.

\item Im BIOS unter dem Eintrag {\tt Advanced Settings} das CDROM Laufwerk als 
  erste Boot-Quelle ausw"ahlen.

\end{itemize}

\end{minipage}


\newpage % %%%%%%%%%%%%%%%%%%%%%%%%%%%%%%%%%%%%%%%%%%%%%%%%%%%%%%%%%%%%%%%

% ========= PAGE 5 ===============
\begin{minipage}[t]{7cm}
\hfill {\Large\sf 5}\\
\hrule

\medskip

\shadowbox{{\sf GNU/Linux: Das Betriebssystem}}

\begin{itemize}
  
\item Freiheit zum Teilen durch freie Nutzungserlaubnis, d.h. das
  Kopieren der GNU/Linux CDs ist legal. Das ist bei Microsoft Windows verboten!

\item Riesige Sammlung an freien Programmen auf einer DVD oder einem Satz
  CDROMs.

\item Kostenlos bzw. sehr niedriger Preis 
\end{itemize}

Zum Ausprobieren empfehle ich {\bf Ubuntu Linux}

\begin{center}
      \url{http://www.ubuntulinux.org}
\end{center}

\begin{itemize}
\item {\bf Live CD}: Linux ausprobieren ohne zu installieren (empfohlen!)
\item {\bf Install CD}: Zur Installation auf der Festplatte
\end{itemize}

CDs mit Ubuntu Linux kann man kostenlos anfordern unter

\bigskip

\centerline{\url{http://www.ubuntulinux.org/shipit}}


% \Ovalbox{
%     See my homepage for more information.
% }

\end{minipage}
\hskip 12mm
%\rule[-18.8cm]{0.2pt}{10mm}
\hskip 12mm
% ========= PAGE 6 ===============
\begin{minipage}[t]{7cm}
\hfill {\Large\sf 6}\\
\hrule

\medskip

\shadowbox{{\sf Falls man doch Windows braucht}}

Dann empfehle ich, Microsoft Windows unter Linux zu installieren. Dazu
braucht man spezielle Software ("`virtuelle Maschine"'), z.B.

\begin{itemize}

\item {\bf Win4Lin}, \url{http://www.win4lin.com}. Kostet 30 Euro bei 
   \url{http://www.ixsoft.de}

\item {\bf VmWare}, \url{http://www.vmware.com}. Kostet etwa 120 Euro f"ur
  Sch"uler und Lehrer bei \url{http://www.ixsoft.de} 

\end{itemize}

Damit kann man Linux und Windows {\em gleichzeitig} verwenden.

\medskip

\shadowbox{{\sf Mehr zum Thema}}

\medskip

Mein VIA EPIA Projekt an der FH:

\centerline{{\tt http://www.fh-augsburg.de/\~{}hhoegl/}}
\centerline{{\tt proj/epia/epia.html}}

\bigskip

F"ur alle:

\smallskip

-- \url{http://www.linux-fuer-alle.de}

-- \url{http://www.linuxeinsteiger.info} 

-- \url{http://www.pro-linux.de} 


\bigskip

Mehr f"ur Lehrerinnen und Lehrer:

\medskip

-- Karl Sarnow, {\em Linux in der Schule}, Millin Verlag 2005

-- {\em Edu Knoppix} - Freie Software und Bildung e.V.
\url{http://www.fsub.org}

-- {\em Skolelinux} -- freie Software f"ur Schulen, 
   \url{http://www.skolelinux.de}

% \vskip 14cm

% \vfill

% {\footnotesize Dieser Text steht unter der {\em Creative Commons
%    Attribution License},
%    \url{http://www.creativecommons.org}. 
% }

\end{minipage}
\hskip 12mm
%\rule[-18.8cm]{0.2pt}{10mm}
% ========= TITLEPAGE ===============
\begin{minipage}[t]{7cm}
\hfill {\Large\sf \ }\\

% \bigskip

{\Large
\begin{center}
   Schluss mit einfachen USB Schnittstellenwandlern
	 als Ersatz f"ur USB:

\smallskip
	USB lernen, verstehen

\smallskip
	
	und anwenden mit

\smallskip

	freien USB Stacks

\end{center} 
}


\vskip 5mm

\centerline{\large\bf www.usb-projects.net}

\vskip 5mm

\bigskip

\centerline{FH Augsburg}

\bigskip

\centerline{Benedikt Sauter}

\bigskip

\centerline{{\tt sauter@ixbat.de}}

\bigskip

\centerline{{\tt http://www.usb-projects.net}}

\end{minipage}
\end{document}


