\documentclass[11pt]{article}

% TODO
%   - "Linux Inside" Stempel


% CV and Expertise

\usepackage{a4}
\usepackage{lscape}
\usepackage{fancybox,graphicx,url,german}

\evensidemargin=0cm
\oddsidemargin=0cm
\hoffset=-0.5cm
\topmargin=-2cm
\textwidth=29.7cm
\textheight=21cm

\pagestyle{empty}
\parindent=0pt

\pdfpageheight=\textheight
\pdfpagewidth=\textwidth

\begin{document}


% ========= PAGE 2 ===============
\begin{minipage}[t]{7cm}
\hfill {\Large\sf 2}\\
\hrule
\shadowbox{{\sf Das Problem}}

\bigskip

\begin{itemize}
\item USB scheint f"ur viele Entwickler sehr kompliziert zu sein 
\item Oft werden deshalb nur einfache Schnittstellenwandler eingesetzt
\item Die Funktinalit�t von USB kann aufgrund desssen nur in einem sehr kleinem Umfang genutzt werden 
\end{itemize}

\medskip

\shadowbox{{\sf Die L"osung}}

\begin{center}
  {\large USB Controller mit freien Stacks verwenden!}
\end{center}

Vorteile:

\begin{itemize}
\item Unabh"angigkeit von USB Bausteinen
\item Nutzung fertiger Treiber
\item Flexibel Reaktionen auf verschiedenste Anforderungen
\item Verf"ugbarkeitn vieler USB Funktionalit"ten (�bertragungsarten, Standbye-Verhalten,...) 
\end{itemize}

Um sich schnell und einfach in USB einarbeiten zu k�nnen,
steht auf http://www.usb-projects.net ein
Grundkurs zur Vermittlung der wesentlichen Eigenschaften.

% \fbox{Just an fbox test}

\end{minipage}
\hskip 12mm
\rule[-18.8cm]{0.2pt}{5mm}  % tickmark
\hskip 12mm
% ========= PAGE 3 ===============
\begin{minipage}[t]{7cm}
\hfill {\Large\sf 3}\\
\hrule

\shadowbox{{\sf USB Stacks}}
\newline
\textbf{USB Device Stack (usbn2mc):}
\begin{itemize}
\item Einfache API
\item Zahlreiche Beispielprojekte
\item Flexible Schnittstelle f"ur Mikrocontroller (ARM7,R8C,AVR) 
\item USB Baustein USBN9404/03 (National Semiconductor)
\item Treiber in der Entwicklung f�r PDIUSBD12D (NXP) und AT90USB (Atmel)
\item http://usbn2mc.berlios.org
\end{itemize}

\textbf{USB Host Stack:}
\begin{itemize}
\item Einfache API
\item Eigene Host Controller Treiber (aktuell SL811HS (Cypress))
\item Fertige Ger"atetreiber (Massenspeichern,HID,HUB)
\item Debug Monitor unterst"utzt die Entwicklung
\item http://www.embedded-projects.net/usbhost
\end{itemize}

\vskip 5mm

\end{minipage}
\hskip 12mm
\rule[-18.8cm]{0.2pt}{5mm}  % tickmark
\hskip 12mm
% ========= PAGE 4 ===============
\begin{minipage}[t]{7cm}
\hfill {\Large\sf 4}\\
\hrule
\shadowbox{{\sf Projekte}}
\newline
\textbf{usbprog, programmierbarer Adapter:}
	\begin{itemize}
	\item Frimware: AVRISP mkII Klon, ISP, RS232, JTAG,...
	\item http://www.embedded-projects.net/usbprog
	\end{itemize}

\textbf{vPort uC Schnittstellen f"ur den PC:}
	\begin{itemize}
	\item Linux Treiber / libusb basierte Bibliothek
	\item Ansteuerbar "uber das Netzwerk mit http://easyconnect.berlios.de
	\item http://www.embedded-projects.net/vport
	\end{itemize}

\textbf{Netzwerkkarte mit TCP/IP Stack:}
  \begin{itemize}
	\item Kommunikation mit einem USB Ger"at "uber Sockets
	\item http://www.ixbat.de/tcp
	\end{itemize}

\textbf{PS2 zu USB Wandler (HID):}
	\begin{itemize}
	\item http://www.ixbat.de/ps2
	\end{itemize}

\textbf{Logik Analysator unter 15 EUR:}
	\begin{itemize}
	\item http://www.ixbat.de/la
	\end{itemize}

\textbf{AVR USB Bootloader:}
	\begin{itemize}
	\item http://www.ixbat.de/boot
	\end{itemize}

\end{minipage}


\newpage % %%%%%%%%%%%%%%%%%%%%%%%%%%%%%%%%%%%%%%%%%%%%%%%%%%%%%%%%%%%%%%%

% ========= PAGE 5 ===============
\begin{minipage}[t]{7cm}
\hfill {\Large\sf 5}\\
\hrule

\medskip

\shadowbox{{\sf USB Debug-Techniken}}

\textbf{Linux, usbmon}
	\newline
	Es gibt in Linux ein Kernelmodul, das beim Debuggen sehr
	hilfreich ist. Zu finden ist es unter Device Driver - USB Support - USB Monitor.
	Mit usbmon kann man sich den Traffic auf dem USB Bus live ansehen.
	\newline
	Vorgehensweise:
		\begin{enumerate}
			\item modprobe usbmon 
			\item mount -t debugfs none\_debugs /sys/kernel/debug
			\item  cat /sys/kernel/debug/usbmon/Xt X = Ger"atenummer (1,2,3..)
		\end{enumerate}
	Beispielausgabe:
		d5ea89a0 3575914555 S Ci:001:00 s a3 00 0000 0003 0004 4 <
	\newline
	Ci: (Control IN Transfer "uber Endpunkt 0 mit Adresse 1), 
	anschliessend folgen die tats"achlichen Daten.
	\newline
	\newline
\textbf{Windows, usbsnoopy}
	\newline
	Mit usbsnoopy kann man "ahnlich wie mit usbmon Traffic aufzeichnen.
	
	Usbsnoopy funktioniert "ahnlich wie usbmon. Jedoch muss man hier
	zuerst aufzeichnen um die Daten analysieren zu k"onnen und kann
	nicht \textit{live} auf den Bus sehen.

	\begin{center}
      \url{http://www.wingmanteam.com/usbsnoopy/}
\end{center}


% \Ovalbox{
%     See my homepage for more information.
% }

\end{minipage}
\hskip 12mm
%\rule[-18.8cm]{0.2pt}{10mm}
\hskip 12mm
% ========= PAGE 6 ===============
\begin{minipage}[t]{7cm}
\hfill {\Large\sf 6}\\
\hrule

\medskip

\shadowbox{{\sf libusb - Systemunabh"angig}}

F"ur den, der auf dem PC betriebssystemunabh"angig entwicklen m"ochte,
ist die Bibliothek libusb eine echte Alternative.
	
\begin{center}
      \url{http://libusb.sourceforge.net/}
\end{center}

Mit libusb kann mit folgenden Betriebssystemen
auf den USB Bus zugreifen:
\begin{itemize}
	\item Linux
	\item Windows (alle aktuellen Versionen)
	\item FreeBSD
	\item NetBSD
	\item OpenBSD
	\item Darwin
	\item MacOS X
\end{itemize}

Dadurch kann man kompatible Anwendungen und Schnittstellen
entwickeln.


\bigskip
\shadowbox{{\sf Konakt}}

\medskip

http://www.usb-projects.net

Benedikt Sauter

sauter@ixbat.de


\end{minipage}
\hskip 12mm
%\rule[-18.8cm]{0.2pt}{10mm}
% ========= TITLEPAGE ===============
\begin{minipage}[t]{7cm}
\hfill {\Large\sf \ }\\

% \bigskip

{\Large
\begin{center}
   Schluss mit einfachen USB Schnittstellenwandlern
	 als Ersatz f"ur USB:

\smallskip
	USB lernen, verstehen

\smallskip
	
	und anwenden mit

\smallskip

	freien USB Stacks.

\end{center} 
}


\vskip 5mm

\centerline{\large\bf www.usb-projects.net}

\vskip 5mm

\bigskip

\centerline{FH Augsburg}

\bigskip

\centerline{Benedikt Sauter}

\bigskip

\centerline{{\tt sauter@ixbat.de}}

\bigskip

\centerline{{\tt http://www.usb-projects.net}}

\end{minipage}
\end{document}


